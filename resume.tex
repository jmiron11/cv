\documentclass[10pt]{article}
\usepackage[T1]{fontenc}
\usepackage[utf8]{inputenc}
\usepackage[margin=0.9in]{geometry}
\usepackage{setspace}
\usepackage{titlesec}
\usepackage{lmodern}
\usepackage[dvipsnames]{xcolor}

\usepackage{hyperref}
\hypersetup{
    colorlinks=true,
    linkcolor=blue,
    filecolor=magenta,      
    urlcolor=MidnightBlue,
}

\newcommand{\HRule}{\rule{\linewidth}{0.5mm}}
\newcommand{\Hrule}{\par\rule{\linewidth}{0.3mm}}

\renewcommand{\baselinestretch}{1.14} 

\title{Curriculum Vitae}
\author{Justin Miron}

\titleformat{\section}{\Large\scshape\raggedright}{}{0em}{}

\titlespacing\section{0pt}{8pt plus 4pt minus 2pt}{0pt plus 2pt minus 2pt}
\setlength{\leftskip}{2pt}
\newcommand{\leftrightrow}[2]{
	#1 \hfill #2 \\
}

\newcommand{\titlesection}{
\parindent=0pt
\setlength{\parskip}{0.1em}

\par{\begin{center}\LARGE Justin Miron\end{center}}

\begin{minipage}[t]{0.40\textwidth}
Address: 110 Lake Street, Ithaca, NY, 14850 \\
Links: \href{http://justinmiron.com}{Website}, \href{https://github.com/jmiron11}{Github}, \href{https://www.linkedin.com/in/justinmiron/}{LinkedIn}
\end{minipage}
\hfill
\begin{minipage}[t]{0.40\textwidth}
Email: \href{mailto:justinmiron@cs.cornell.edu}{justinmiron@cs.cornell.edu} \\
Phone: (810) 841-8557
\end{minipage}
\vspace{0.3em}
\Hrule
}

% \newcommand{\teachblock}[2]{
% 	\leftrightrow{#1}{#2}
% }

\newcommand{\teachblock}[2]{
\begin{minipage}[t]{0.20\textwidth}
\vspace*{0.4em}
\begin{flushright}#2\end{flushright}
\end{minipage}
\hfill\vline\hfill
\begin{minipage}[t]{0.76\textwidth}
\vspace*{0.4em}
	#1
\end{minipage}
\vspace{0.2em}
}

\newcommand{\workblock}[4]{
\begin{minipage}[t]{0.14\textwidth}
\vspace*{0.4em}
\begin{flushright}#3\end{flushright}
\end{minipage}
\hfill\vline\hfill
\begin{minipage}[t]{0.82\textwidth}
\vspace*{0.4em}
	\textbf{#1} \\
	#2 \\
    #4
\end{minipage}
\vspace{0.7em}
}

\newcommand{\workblockempty}[2]{
\begin{minipage}[t]{0.14\textwidth}
\vspace*{0.4em}
\begin{flushright}#2\end{flushright}
\end{minipage}
\hfill\vline\hfill
\begin{minipage}[t]{0.82\textwidth}
\vspace*{0.4em}
	\textbf{#1}
\end{minipage}
\vspace{0.1em}
}

\newcommand{\workblocktwo}[3]{
\begin{minipage}[t]{0.14\textwidth}
\vspace*{0.4em}
\begin{flushright}#2\end{flushright}
\end{minipage}
\hfill\vline\hfill
\begin{minipage}[t]{0.82\textwidth}
\vspace*{0.4em}
	\textbf{#1} \\
    #3
\end{minipage}
\vspace{0.7em}
}


\newcommand{\researchblock}[3]{
	\leftrightrow{\textbf{#1}}{#2}#3
}


\newcommand{\educationblock}[3]{
	\leftrightrow{\textbf{#1}}{#2}
    #3
}

% \newcommand{\educationblock}[3]{
% \begin{minipage}[t]{0.14\textwidth}
% \vspace*{0.4em}
% \begin{flushright}#2\end{flushright}
% \end{minipage}
% \hfill\vline\hfill
% \begin{minipage}[t]{0.82\textwidth}
% \vspace*{0.4em}
% 	\textbf{#1} \\
% 	#3 \\
% \end{minipage}
% \vspace{0.7em}
% }

\begin{document}
\titlesection
\setlength{\parskip}{0em}
\vspace{-2.5em}
\section{Education}
\vspace{0.1em}
\educationblock{Ph.D. Student, Cornell University}{August 2017 - }{Department of Computer Science \\
Advisor: \href{http://www.cs.cornell.edu/~ragarwal/}{Prof. Rachit Agarwal}}
\vspace{0.5em}

\educationblock{B.S., University of Illinois at Urbana-Champaign}{August 2013 - May 2017}{Department of Computer Engineering}

\section{Relevant Coursework}
\vspace{0.1em}
Graduate \{Computer Networks, Distributed Algorithms, Distributed Systems, Computer Networks, Programming Languages \& Compilers\}, Computer Architecture, Networking with Big Data, Operating System Design

\section{Work Experience}
\vspace{-0.4em}
\workblocktwo{Research Assistant at Cornell, Ithaca, NY}{Fall 2017-}{
Designing an encrypted key-value store that hides access patterns from server-side adversaries through
new distributed systems and encryption techniques.
Co-designing scheduling and caching in analytical batch processing systems to improve server utilization and performance through multiplexing cache resources across the compute layer.}

\workblock{Software Engineering Intern at Google, Sunnyvale, CA}{\href{https://ai.google/research/pubs/pub43838}{Network Infrastructure}}{Summer 2018}{
Built a framework (enabled in production) for pluggable bandwidth allocation policies for fine-grained groups of flows on Google's inter-datacenter WANs. Designed new techniques on top of the framework to improve allocation through machine learning and real-time control system algorithms; and algorithmic analysis of existing bandwidth allocation techniques.
}

\workblock{Software Engineering Intern at Microsoft, Aliso Viejo, CA}{\href{https://azure.microsoft.com/en-us/services/sql-data-warehouse/}{Azure Data Warehouse Group}}{Summer 2017}{
Built a distributed query monitoring system for Azure Data Warehouse through instrumentation and communication across distributed query execution components. This enabled understanding query execution across disparate components for easier debuggability and performance profiling.}

\workblocktwo{Software Engineer at \href{http://www.charmplusplus.com/}{Charmworks Inc.}, Champaign, IL}{2017}{
Modified shared memory parallel applications built for high performance clusters to operate on distributed memory runtimes (executed on the cloud) with low overhead. Characterized and tackled challenges involved with overcoming higher cloud latencies when moving scientific computing applications to cloud.}

\workblocktwo{Research Assistant at University of Illinois at Urbana-Champaign, Urbana, IL}{2017}{
	Co-designed algorithms for load balancing and replication of compute objects to
	quickly respond to load imbalance in clusters. Lowered latency of communication protocols for a parallel runtime system through a no-copy RDMA protocol, removing a data copy from the end-to-end latency, and a new design for a near-unbounded multi-producer multi-consumer queue that lead to a 60\% improvement over existing lock-free approaches).
}

\workblockempty{Intern with \href{https://cloud.google.com/compute/}{Google Compute Engine} at Google, Seattle, WA}{Summer 2016}

\workblockempty{Intern at ViaSat Inc., Carlsbad, CA}{Summer 2015}

\vspace{-0.4em}

\section{Side-Projects}
\textbf{Programmable caching} (\textit{C++, Bazel, gRPC}): A framework for easily programming a cluster caching layer to implement flexible and dynamic caching policies. \\
\textbf{Never stop bouncing} (\textit{C, SDL, CMake}): A 2D side-scroller game built in SDL about a bouncy ball trying to find true love. \\
\textbf{Web-scraping Detroit Red Wing statistics} (\textit{Python}): Using web-scraped statistics to show that the Detroit Red Wings are the single greatest hockey team. \\
\textbf{Music-swap} (\textit{Android, Java}): Android app that allows users to match according to shared music tastes and exchange artists (pretty much tinder for music).

\vspace{-0.4em}

\section{Relevant Teaching Experience}
\vspace{-0.8em}
\teachblock{\textbf{Computer Networks}, Cornell University, \textit{Teaching Assistant}}{Jan 2018 - May 2018} \\
\teachblock{\textbf{Database Systems}, Cornell University, \textit{Teaching Assistant}}{Aug 2017 - Dec 2017} \\
\teachblock{\textbf{Data Structures}, University of Illinois at Urbana-Champaign, \textit{Course Assistant}}{Aug 2015 - May 2017}

\vspace{-0.44em}

\section{Programming Skills}
\vspace{0.1em}
\textbf{Languages:} C, C++, Python, Java \\
\textit{Familiar with various parallel computing, RPC, build, and version control frameworks/systems.}

\end{document}
